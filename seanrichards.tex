\section{Liouville's Theorem for the Gifted Amateur}
\subsection*{Sean Richards}

Liouville's Theorem has profound implications for Classical and 
Statistical mechanics. Formally, we write

$$\frac{d\rho}{dt}=\frac{\partial\rho}{\partial t}+\{\rho, 
\mathcal{H}\}$$

Where we identify the final term as the \textit{Poisson Bracket} of 
$\rho$ and $\mathcal{H}$, and $\rho$ is the density function of the 
arbitrary ensemble in question. Heuristically, we say that the volume of 
phase space remains constant, regardless of the trajectory taken through 
it.

Naturally, as each point in phase space is identified with a different 
state of a system, Liouville's Theorem is essential for calculations in 
Statistical Mechanics; by considering Hamilton's equations of motion at 
any time $t$ (in the future or the past), one may calculate the state of 
the system at that time.

Given that each point in phase space is a state of the system; and given 
that the volume of phase space is finite, we may identify the volume as 
a set of all possible states permissible to a system. Liouville's 
Theorem now says that if we take a different trajectory through phase 
space, the number of states that a system may inhabit stay constant. 
Hence the same analysis that determines one volume of space can be 
performed to determine the other.
