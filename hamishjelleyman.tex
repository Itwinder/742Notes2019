\subsection{By Hamish Jelleyman}
To consider Liouville’s theorem, it is nice to start with an example. Take a simple harmonic oscillator like a mass on a spring. We can set the mass in motion by pulling on it and letting go, thus giving it an initial position and an initial momentum. The motion of the mass can be calculated in a number of ways from this point. One of which is using Hamilton’s equations. The mass will trace out a path in phase space as its position and momentum changes. Now consider we do this three more times, each time giving a different initial momentum and position to our mass. The positions in phase space of the different oscillating masses can be joined together to form a quadrilateral of some area $A$.\medskip 


Liouville’s theorem states that as the system evolves, this area between the four points will remain constant. We can generalise this even further to say that for a distribution of many particles the density of particles in momentum-position phase space will remain constant as a system evolves. \medskip 


In the context of a macroscopic state of a system. Each macrostate will be associated with a number of microstates with different phase-space initial conditions which will satisfy $H(p,q,t)=E$. By considering the density function of these initial conditions and applying Hamiltonian dynamics on it we can gain understanding on how the macrostate is likely to evolve without having to simulate the motion of every individual particle.
